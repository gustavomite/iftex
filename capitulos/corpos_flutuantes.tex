Corpos flutuantes são elementos não textuais como figuras e tabelas que complementam as informações do texto. Neste capítulo são expostos breves exemplos dos corpos flutuantes disponíveis na classe {IF\TeX}.

Na Seção \ref{secao:figuras} é mostrado como inserir figuras, a Seção \ref{secao:tabelas_e_quadros} explica como incluir tabelas e quadros e a Seção \ref{secao:algoritmos_e_codigos} demostra como trabalhar com algoritmos e códigos fontes.

\section{Figuras}
\label{secao:figuras}

A inserção de figuras é realizada normalmente através do comando \comando{begin\{figure\}}. A Figura \ref{figura:logomarca_if} exibe a logomarca do IF. De acordo com as normas ABNT a lista de figuras é um elemento opcional do documento, para incluí-la é preciso inserir o comando \comando{inserirlistafiguras} antes do início do documento.

\begin{figure}[htb]  \centering
 \iflogo
 \caption{Logomarca do IF} \label{figura:logomarca_if}
\end{figure}

\section{Tabelas e Quadros}
\label{secao:tabelas_e_quadros}

A inserção de tabelas e quadros é feita de forma semelhante a inserção de figuras, porém são utilizados os ambientes \textit{table} e \textit{quadro}. A principal diferença entre tabelas e quadros, de acordo com \citet{castro:2016:manual}, é que as tabelas são destinadas para informações numéricas e os quadros são mais adequados para informações textuais.

Como exemplos foram inseridas a Tabela \ref{tabela:lista_produtos} que exibe uma de lista de produtos e a Tabela \ref{tabela:populacao_america_sul} que mostra a população dos países da América do Sul. Foi inserido também o Quadro \ref{quadro:editores_texto_livres} com alguns editores que podem ser usados para se trabalhar com Latex para demonstrar a inserção de quadros.

 A lista de tabelas também é um elemento opcional que pode ser incluída com o comando \comando{inserirlistatabelas} antes do início do documento. O mesmo acontece com a lista de quadros que pode ser incluída com o comando \comando{inserirlistaquadros}.

\begin{table}[htb] \centering
\caption{Lista de produtos} \label{tabela:lista_produtos}
\begin{tabularx}{\textwidth}{X|l|r|r|r} \hline
Produto      & Unidade & Preço (R\$) & Quantidade & Total (R\$) \\ \hline
Arroz        & Kg      & 2,00        & 550        & 1.100,00    \\
Óleo de Soja & L       & 2,50        & 500        & 750,00      \\
Açucar       & Kg      & 3,00        & 100        & 300,00      \\ \hline
\end{tabularx}
\end{table}

\begin{table}[htb] \centering
\caption{População dos países da América do Sul} \label{tabela:populacao_america_sul}
\begin{tabular}{r|l|r}        \hline
Código  & País            & População   \\ \hline
1       & Brasil          & 191.480.630 \\
2       & Argentina       &  39.934.100 \\
3       & Colômbia        &  46.741.100 \\
4       & Paraguai        &   9.694.200 \\
5       & Uruguai         &   3.350.500 \\
6       & Peru            &  28.221.500 \\
7       & Equador         &  13.481.200 \\
8       & Bolívia         &   9.694.200 \\
9       & Venezuela       &  28.121.700 \\
10      & Chile           &  16.803.000 \\ \hline
\end{tabular}
\legend{Fonte: \citet{wikipedia:2011:america_sul}.}
\end{table}

\begin{quadro}[htb] \centering
\begin{tabular}{|l|l|r|}        \hline
Editor     & Multiplataforma & Específico para Latex \\ \hline
Kwriter    & Sim             & Não                   \\
Texmaker   & Sim             & Sim                   \\
Kile       & Sim             & Sim                   \\
Geany      & Sim             & Não                   \\ \hline
\end{tabular}
\caption{Editores de Texto Livres} \label{quadro:editores_texto_livres}
\end{quadro}

\section{Algoritmos e Códigos} \label{secao:algoritmos_e_codigos}

Além dos corpos flutuantes convencionais para inserir figuras (\comando{begin\{figure\}}) e tabelas (\comando{begin\{figure\}}), a classe {IF\TeX} possui mais dois tipos de corpos flutuantes um para algoritmos (\comando{begin\{algoritmo\}}) e outro para códigos (\comando{begin\{codigo\}}). Como exemplo temos o Algoritmo \ref{algoritmo:mdc} que calcula o máximo divisor comum entre dois números e os Códigos \ref{codigo:notas_alunos} e \ref{codigo:metodo_leitura} que são uma consulta na \textit{Structured Query Language (SQL)} e um método em Java que recebe um texto digitado pelo usuário, respectivamente.

\begin{algoritmo}[!htb]
\begin{algorithmic}[1]
 \Require Dois números inteiros ($n_1, n_2$)
 \If{$n_2 > n_1$} \Comment{Garante que o maior número seja $n_1$}
   \State troca valores de $n_1$ e $n_2$
 \EndIf
 \Repeat
   \State $r \leftarrow$ resto da divisão de $n_1$ por $n_2$
   \State $n_1 \leftarrow n_2$
   \State $n_2 \leftarrow r$
 \Until{$r > 0$}
 \Return $n_1$
\end{algorithmic}
\caption{Algoritmo para cálculo de máximo divisor comum MDC($n_1$,$n_2$)} \label{algoritmo:mdc}
\end{algoritmo}

\begin{codigo}[!htb]
\begin{Verbatim}
SELECT a.nome_aluno AS aluno,
       d.nome_disciplina AS disciplina,
       m.nota AS nota
FROM aluno AS a,
     disciplina AS d,
     matriculado AS m
WHERE a.id_aluno = m.id_aluno
  AND d.id_disciplina = m.id_disciplina
ORDER BY a.nome_aluno, d.nome_disciplina;
\end{Verbatim}
\caption{Consulta SQL} \label{codigo:notas_alunos}
\end{codigo}

\begin{codigo}[!htb]
\begin{Verbatim}
public static String Leitura(){
    BufferedReader reader =
        new BufferedReader(new InputStreamReader(System.in));
    try {
        return reader.readLine(); // Lê uma linha pelo teclado
    } catch (IOException e) {
        e.printStackTrace();
        return "";
    }
}
\end{Verbatim}
\caption{Sub-rotina para obter uma entrada do usuário} \label{codigo:metodo_leitura}
\end{codigo}

Existem diversos outros pacotes disponíveis para escrever algoritmos e códigos. Nos exemplos anteriormente foram utilizados o pacote \textit{algpseudocode} e \textit{fancyvrb}. O pacote \textit{algpseudocode} é usado para escrever algoritmos em alto nível \cite{janos:2005:algpseudocode}. Já o pacote \textit{fancyvrb} serve para escrever códigos mono-espaçados \cite{zandt:2010:fancyvrb}.
Caso sejam utilizados os ambientes de algoritmo e código, podem ser incluídos os comandos \comando{inserirlistaalgoritmos} e \comando{inserirlistacodigos} antes do \comando{begin\{document\}} para que a lista de algoritmos e a lista de código sejam criadas.
Existem também diversos outros pacotes para formatação de algoritmos e códigos que podem ser usados como o \textit{minted} com suporte a diversas linguagens de programação \cite{poore:2016:minted}.
