Este documento explica brevemente como trabalhar com a classe \textit{Latex} {IFMG\TeX} para confeccionar trabalhos acadêmicos seguindo as normas da Associação Brasileira de Normas Técnicas (ABNT) e o \textit{Manual de Normalização para Apresentação de Trabalho de Conclusão de Curso} do Instituto Federal Minas Gerais (IFMG) - Campus Bambuí \cite{castro:2016:manual}.
O referido manual foi desenvolvido com o intuito de padronizar as trabalhos acadêmicos produzidos na instituição.

A classe {IFMG\TeX} foi construída com base na classe \textit{abntex2} mantendo as mesmas opções presentes nesta classe, portanto é recomendável que seja consultada a documentação da mesma \cite{araujo:2016:abntex2}.
A classe \textit{abntex2} foi desenvolvida para facilitar a escrita de documentos seguindo as normas da ABNT.
O requisito básico para utilização da classe {IFMG\TeX} é criar um documento com o comando \comando{documentclass\{ifmgtex\}}.
Por padrão, a classe {IFMG\TeX}, cria um documento frente e verso.
Para documentos somente com anverso, é necessário informar a opção \textbf{onseside} (comando \comando{documentclass[oneside]\{ifmgtex\}}).
