\documentclass{apostila}

\title{Template de Apostila}
\author{Gustavo Miranda Teixeira}
\date{Abril 2020}
\instituicao{Instituto Federal Minas Gerais}
\unidade{Campus Rio Pomba}
\curso{Bacharelado}{Ciência da Computação}
\capaimg{capa.png}

\begin{document}


% -----------------------------------------------
% Gera elementos pré-textuais
% -----------------------------------------------
\maketitle

% -----------------------------------------------
% Capítulo: Introdução
% -----------------------------------------------
\chapter{Introdução}

Deve-se usar o comando \textbf{\textbackslash chapter} para dividir o conteúdo da apostila em aulas ou capítulos. Cada capítulo poderá conter um número definido de seções utilizando o comando \textbf{\textbackslash section}.

\section{Primeira Seção}
Cada capítulo poderá ser subdividido em uma ou mais seções.

\section{Segunda Seção}
Cada seção poderá ser subdividido em uma ou mais subseções. Para isso, basta utilizar novamente o camndo \textbf{\textbackslash section} para cada nova seção criada.

\subsection{Subseção}
Para criar uma subseção, utiliza-se o comando \textbf{\textbackslash subsection}.

\chapter{Listas}
Além de divisões em capítulos, seções e subseções, é possível utilizar listas de itens enumeradas, ou não.

\section{Itens}

Podemos criar uma lista de ítens utilizando o comando \textbf{\textbackslash begin\{itemize\}}:
\begin{itemize}
 \item Cada item deve ser identificado com o comando \textbf{\textbackslash item};
 \item Podemos adicionar quantos itens quisermos;
 \item Após o último item, é necessário indicar o encerramento da lista com o comando \textbf{\textbackslash end\{itemize\}}.
\end{itemize}

\section{Itens Enumerados}

Listas também podem ser criadas utilizando uma numeração para seus items. Nesse caso, ao invés de iniciar um \textit{itemize}, vamos utilizar um \textit{enumerate}: \textbf{\textbackslash begin\{itemize\}}:
\begin{enumerate}
 \item A lista deve ser iniciada com o comando \textbf{\textbackslash begin\{enumerate\}};
 \item Cada item deve ser identificado com o comando \textbf{\textbackslash item};
 \item Após o último item, o comando \textbf{\textbackslash end\{enumerate\}} é usado para encerrar a lista.
\end{enumerate}

% -----------------------------------------------

\chapter{Considerações Finais}

Essa apostila é baseada no modelo de Marcos R. Ribeiro \cite{marcos:site} com pequenas adaptações.

% -----------------------------------------------
% Elementos pós-textuais
% -----------------------------------------------
\postextual

% -----------------------------------------------
% Referências bibliográficas
% -----------------------------------------------
\nocite{araujo:2016:abntex2}
\nocite{marcos:site}
\bibliography{referencias}



\end{document}
